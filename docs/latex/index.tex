Може да видите {\itshape документацията} на кода, генерирана от {\bfseries Doxy\+Gen} \href{https://ivanov1234159.github.io/JurassicPark/html/index.html}{\tt тук}

{\bfseries Автор\+:} Веселин Иванов

\subsection*{Описанире на проекта}

Ванката беше изключително впечатлен от последната лекция в историческия музей и има нова хрумка -\/ ще става палеонтолог! Не, още по-\/добре, ще си отвори ферма за динозаври! Въпреки безумните убеждения на майка си, Ванката вече има план как ще изглежда новият домашен зоопарк за праисторически влечуги\+: динозаврите могат да са няколко вида\+:


\begin{DoxyItemize}
\item месоядни
\item тревопасни
\item водни
\item летящи
\end{DoxyItemize}

от книгите знае, че те не съжителстват заедно, затова трябва да са разделени

необходимо е да има достатъчно храна за всички

необходимо е да има достатъчно персонал в парка, за да се грижат за всички.

{\itshape Вашата основна задача е да наподобите модела на успешен развъдник на динозаври.}

Всеки {\bfseries динозавър} се характеризира чрез\+:


\begin{DoxyItemize}
\item име
\item пол
\item ера
\begin{DoxyItemize}
\item Триас
\item Креда
\item Юра
\end{DoxyItemize}
\item разред
\begin{DoxyItemize}
\item тревопасен
\item месояде
\item летящ
\item воден
\end{DoxyItemize}
\item вид
\begin{DoxyItemize}
\item плезиозавър
\item бронтозавър
\item тиранозавър
\item и т.\+н.
\end{DoxyItemize}
\item храна
\begin{DoxyItemize}
\item трева
\item месо
\item риба
\end{DoxyItemize}
\end{DoxyItemize}

Те са разпределени в клетки, като заедно могат да съжителстват единствено животни от една ера и един разред (но не задължително от един вид).

Важна информация за {\bfseries клетката} е\+:


\begin{DoxyItemize}
\item размер
\begin{DoxyItemize}
\item малка (за 1 животно)
\item средна (до 3 животни)
\item голема (до 10 животни)
\end{DoxyItemize}
\item климат
\begin{DoxyItemize}
\item сухоземен
\item въздушен
\item воден
\item и т.\+н.
\end{DoxyItemize}
\item животни, които я обитават
\item ера на животните вътре (ако има такива)
\end{DoxyItemize}

В нашия зоопарк отначало има произволно количество клетки и без никакви животни. Ново животно може да се приеме, ако има подходяща клетка за него с празно място в нея. Ако няма, то такава може да бъде построена, посочвайки съответния климат и размер.

\subsubsection*{Напишете система за управление на зоопарк, която може да\+:}


\begin{DoxyItemize}
\item приема ново животно
\end{DoxyItemize}


\begin{DoxyCode}
add <name> <gender> <era> <type> <species> <food>
\end{DoxyCode}



\begin{DoxyItemize}
\item строи нова клетка
\end{DoxyItemize}


\begin{DoxyCode}
create <climate> <size>
\end{DoxyCode}



\begin{DoxyItemize}
\item премахва вече налично животно
\end{DoxyItemize}


\begin{DoxyCode}
remove <name>
\end{DoxyCode}



\begin{DoxyItemize}
\item зарежда склада с храна
\end{DoxyItemize}


\begin{DoxyCode}
supply <food\_name> <amount>
\end{DoxyCode}



\begin{DoxyItemize}
\item нахранва всички животни
\end{DoxyItemize}


\begin{DoxyCode}
feed
\end{DoxyCode}



\begin{DoxyItemize}
\item наема нов персонал
\end{DoxyItemize}


\begin{DoxyCode}
hire <count>
\end{DoxyCode}



\begin{DoxyItemize}
\item запазва сегашното състояние на програмата
\end{DoxyItemize}


\begin{DoxyCode}
save
\end{DoxyCode}



\begin{DoxyItemize}
\item показва сегашното състояние на програмата
\end{DoxyItemize}


\begin{DoxyCode}
status
\end{DoxyCode}



\begin{DoxyItemize}
\item показва информация за възможностите на приграмата
\end{DoxyItemize}


\begin{DoxyCode}
help
\end{DoxyCode}



\begin{DoxyItemize}
\item затваряне на приграмата
\end{DoxyItemize}


\begin{DoxyCode}
exit [ <save> = Yes ]
\end{DoxyCode}


\subsection*{Забележки\+:}


\begin{DoxyItemize}
\item Може ли в началото клетките за динозаврите да са наистина случайно генерирани и като брой, и като вид?
\item Не искаме при затваряне на програмата информацията да се губи.
\item Летящите динозаври са в по-\/голямата си част хищни \+:)
\end{DoxyItemize}

\subsection*{}

\subsection*{Външни източници}


\begin{DoxyItemize}
\item \href{https://github.com/onqtam/doctest}{\tt doctest} -\/ за тестването на проекта (програмата) 
\end{DoxyItemize}